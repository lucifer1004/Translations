\chapter*{前言}
\addcontentsline{toc}{chapter}{前言}

任何一本关于诗歌的书的作者都必然抱有这样的期望:它能够让新的心灵感受到阅读诗歌的乐趣,并从中获益。即使从没有读过关于诗歌的书,也并不妨碍一个人欣赏诗歌;对于诗歌的品味常常是与生俱来的,正如写诗的冲动一样。但就像Bacon说的:“天赋有如自然植物,需要仰仗学习以裁剪。”像这样对天赋的培育,正是批评的任务所在。这并非是要为诗歌的好坏设定标准,也并非H. G. Wells在一本小说里描绘的大学教授:“他是教授我们如何欣赏诗歌和散文的老师中的一个,他告诉我们应该在什么时候喊出‘哦’、‘啊’,应该在什么时候悻然地摇摇头,就像一个收到假币的公交车售票员。”对于任何一个热爱诗歌的批评家来说,他的目标必定是让阅读诗歌成为一种探索,持续不断地向读者揭示对于他自己、作者,以及诗歌写作本身的新灵感。这必定是一种邀请:让人去欣赏,去聆听,去在诗歌的存在里徜徉,感受诗歌的魅力。

并且,因为谈论诗歌的唯一目的就是说“过来这里,读它”,这本书里包括了很多首诗;来自众多作者,有经典也有新作,有广为人知的也有罕为听闻的。
