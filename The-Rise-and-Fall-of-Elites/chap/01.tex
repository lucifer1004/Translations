\chapter{一些社会学定律}

本研究旨在借助事实,客观地反驳某些社会学理论。

在社会学或政治经济学方面舞文弄墨的人,其内心往往都有某种必须要维护的实际秩序。在这里,我不想对这一倾向提出谴责,而只想提醒读者我并非如此;我必须作出这一陈述,因为在实际中,作者的言论往往会被阐释得比作者原本意图的更为宽泛。因此,如果一个作者描述了秩序A的某些缺陷,他就会被理所当然地认为是在声讨整个秩序A,并且常常会更进一步,他会被认为偏好处于秩序A的对立面的某种秩序B。

举个例子:如果一个人对普选制度提出疑议,那么他就会被认为偏好受限制的选举;对民主的坏处提出责难的人,就会被认为偏好独裁形式的政府;对君主制的某些方面提出赞扬的人,毫无疑问会被认为是反共和制的,反之亦然;简而言之,每一个字面上特定的陈述都会被赋予一个普遍的含义。这样做并不完全是错误的;甚至,这样做往往是一针见血的,因为一些作者会刻意地少说一些,以期别人会更加相信他——这在文学中是一种值得嘉许的方法,但在科学中则要大打折扣。因此,我认为非常有必要强调这篇文章中的每一句陈述都不包含任何超出它字面意思的含义,也一定不能被在更加宽泛的意义下进行解读。

我还得再用几句话来解释为什么我选择使用当下的事实,而非限定在历史事实的范围内。后者当然也有其优势:人们可以用更为冷静的心智,带着更少的情绪和偏见去审视它们。但它们有一个非常巨大的劣势:我们对它们的了解是非常不全面的,同时,前面提到的优势常常只存在于设想之中,因为我们习惯于将当下的情绪传递到过去。比如说,一个狂热地着迷于日耳曼帝国的德国历史学家,绝不会忍受对于凯撒(Caesar)或奥古斯都(Augustus)的任何诋毁,而一个民主主义者必定会与阿里斯托芬争论得不可开交。

现在让我们进入正题。我们首先列出一些从事实中推导得出的社会学定律,而我们将会用这些事实来反驳它们。这里我们沿用了克劳德·伯纳德(Claude Bernard)推荐的方法,也就是从事实到概念,再从概念回到事实。在这一版本中,读者只会看到第二部分。第一部分,也是长得多的那一部分在我正在撰写的社会学专著中将不再会被省略——如果我能将它完成并出版的话。现在我们暂且把这些定律当成多少有些道理的假说,然后我们会看看能不能在它们的帮助之下成功地解释事实。

首先,我们要注意到人类行为更伟大的部分源于情感,而非逻辑推理。这对于非经济驱动的行为基本上是正确的。而对于经济行为,特别是那些与工商业有联系的经济行为,情况可能恰好相反。人虽然常常受到非逻辑冲动的驱使,但也倾向于从逻辑上把他的行为与特定的原则联系起来;于是他就会事后发明出这些原则,以证明行为的合理性。所以会发生这样的事情:行为A实际上是原因B的结果,但却被作者说成是另一个原因C的结果,而这个原因常常是虚构出来的。

由此可见,每个社会学现象都有两种不同,并常常是迥然相异的形式:一种是客观形式,它决定了真实对象之间的关联;另一种是主观形式,它决定了心理状态之间的关联。就像一面哈哈镜:现实中直的看上去成了弯的,小的成了大的,反之亦然。类似的,人的意识也是这样反映着客观现象,这些现象从历史中,或从当代的见证中进入了我们的知识。因此,如果我们希望知道客观现象,重要的是我们不能局限在主观现象中,而是从后者正确地推导出前者。这实质上就是历史批评的任务,也即从简单的对原材料的批评,提升到对人类心灵的批评。

出于对波斯入侵的恐惧,雅典人派出信使去德尔菲(Delphi)请示神谕。神谕里说了很多事情,其中就包括宙斯曾经应允给特里托格内亚(Tritogenia)一座堵不可征服的木墙。在这之后,雅典人修整了他们的舰队并在萨拉米斯(Salamis)取得了胜利。这是当代的很多人对这一现象的认识,它是希罗多德(Herodotus)留给我们的。但显然它的客观形式是完全不同的。希望在今天已经没有人信仰阿波罗(Apollo)、雅典的特里托格内亚或是宙斯(Zeus)了;这样一来,就必须要找到另外的、更加真实的原因来解释萨拉米斯的胜利;实际上,这场胜利来自于地米斯托克利(Themistocles),他说服了雅典人把国库的资金投资到舰队上。但值得注意的是,希罗多德在对事实的陈述中,并没有介入进来提示这一真正的原因。由于幸运的巧合,刚好有了船,所以很容易就能遵循神谕。按照作者的描述,雅典人仅仅是在阿波罗回应的真正含义上众说纷纭,不知道应该选择哪条道路;有人相信木墙实际上是石墙,而其他人则认为神是在提示舰队。地米斯托克利自己——再一次地,按照希罗多德所说——单纯地讨论了对神谕语言的解读。因此,实际和主观现象之间的对比就更加明显地体现出来了。

然而,仅仅是研究这两种现象以及它们的相互关系是不够的;第三个问题出现了:真正的现象是如何作用在主观现象之上并修改了它的?反过来呢?达尔文主义对这一问题给出了一个非常简单的回答,但不幸的是这一回答只部分正确。根据它的教条,这两种现象之间的关系可以通过对不符合这一关系的个体的逐步排除得到。

在我们的例子里,完全不涉及排除,并且,我们永远也不会知道雅典人到底为什么选择了对神谕的这种解读,而不是另一种;我们也不会知道地米斯托克利的言语是不是听起来非常真挚诚恳。现在,当相似的事情发生,没有绝对的信仰,也没有绝对的不信仰;因此,如果可以用现在的人来衡量当时的人的话,我们会倾向于相信真正的原因,也就是雅典的海军力量,在潜意识里影响了地米斯托克利;在这一冲动之下,他首先说服了自己,然后说服了其他人,神指的是舰队。

我们选的这个例子对一些人来说可能看起来不太合适,因为它太过明显了。但如果我们选了一些和这个古代的例子基本一致的更为现代的例子,那些这样想的人可能很快就会改变想法了\footnote{甚至不一定要是现代的例子;只要现代信念在其中占有一席之地即可。因此,在讨论君士坦丁改宗(Conversion of Constantine)时,布瓦西埃(Boissier,译者注:应当是指法国学者Marie-Louis-Antoine-Gaston Boissier)敢于这样说:“因此,优西比乌(Eusebius)的话的第一部分是非常有可能的……至于另一部分,那是幻象,是梦境,对此我不置一词;这些奇迹般的事件无法批评,也不属于历史学的合理范畴。每个人都可以相信他喜欢的东西,不管优西比乌说的那些是不是真正的事实;如果那确实是的话,那么我们就是见到了真的神迹了……”\parencite[p.~39]{boissier_fin_1894}这真是精妙绝伦!当一个作者讲述寓言或奇迹时,历史学家应当保持尊重的沉默,因为“这样的事件无法批评,也不属于历史学的合理范畴”!\\
    \setlength{\parindent}{2em}
    \indent
    但如果我们不能对君士坦丁感受到的神奇幻象提出质疑,为什么我们应当被允许质疑萨拉米斯的希腊舰队撤退时出现的,那个警告希腊人“你们这些可悲的人!你们打算把你们的船藏起来多久?”的女幽灵?就我个人而言,比起希罗多德,我不觉得有必要给优西比乌更多的信任;但恰恰是布瓦西埃所声称的不能进入这里的批评让我相信:总的来说,优西比乌的故事要比希罗多德的更为可信。}
。在法国,有太多人会说“不朽的1789年原则”或是“捍卫共和”,而在其他国家,很多人会说“捍卫伟大的君主制”——就像地米斯托克利对神谕的解读一样——他们由此给他们的行为赋予了虚构的原因,而掩藏了真正的原因。常言道,一个人能看见同伴眼中的尘埃,却看不见自己眼中的梁木\footnote{译者注:出自《新约·马太福音》7:1-5。}。此言非虚。嘲笑古代迷信的人,往往只不过是用那些并不比他所拒斥的那些东西更合理或更真实的现代版本替换了它们。

现在让我们转向一些没那么众所周知的事情。后面我们会把它们和前面提到的那些结合起来。
